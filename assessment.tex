\documentclass[a4paper, 12pt]{scrartcl}
  % Headers and footers.
  \usepackage{fancyhdr}
  % Enables the use of colour.
  \usepackage{xcolor}
  % Syntax high-lighting for code. Requires Python's pygments.
  \usepackage{minted}
  % Enables the use of umlauts and other accents.
  \usepackage[utf8]{inputenc}
  % Diagrams.
  \usepackage{tikz}
  % Settings for captions, such as sideways captions.
  \usepackage{caption}
  % Symbols for units, like degrees and ohms.
  \usepackage{gensymb}
  % Latin modern fonts - better looking than the defaults.
  \usepackage{lmodern}
  % Allows for columns spanning multiple rows in tables.
  \usepackage{multirow}
  % Better looking tables, including nicer borders.
  \usepackage{booktabs}
  % More math symbols.
  \usepackage{amssymb}
  % More math fonts, like \mathbb.
  \usepackage{amsfonts}
  % More math layouts, equation arrays, etc.
  \usepackage{amsmath}
  % More theorem environments.
  \usepackage{amsthm}
  % More column formats for tables.
  \usepackage{array}
  % Adjust the sizes of box environments.
  \usepackage{adjustbox}
  % Better looking single quotes in verbatim and minted environments.
  \usepackage{upquote}
  % Better blank space decisions.
  \usepackage{xspace}
  % Better looking tikz trees.
  \usepackage{forest}
  % URLs.
  \usepackage[hidelinks]{hyperref}
  % Plotting.
  \usepackage{pgfplots}
  % Calculates the number of pages.
  \usepackage{lastpage}
  % Styling the abstract.
  \usepackage{abstract}

  
  % Various tikz libraries.
  % For drawing mind maps.
  \usetikzlibrary{mindmap}
  % For adding shadows.
  \usetikzlibrary{shadows}
  % Extra arrows tips.
  \usetikzlibrary{arrows.meta}
  % Old arrows.
  \usetikzlibrary{arrows}
  % Automata.
  \usetikzlibrary{automata}
  % For more positioning options.
  \usetikzlibrary{positioning}
  % Creating chains of nodes on a line.
  \usetikzlibrary{chains}
  % Fitting node to contain set of coordinates.
  \usetikzlibrary{fit}
  % Extra shapes for drawing.
  \usetikzlibrary{shapes}
  % For markings on paths.
  \usetikzlibrary{decorations.markings}
  % For advanced calculations.
  \usetikzlibrary{calc}
  
  % GMIT colours.
  \definecolor{gmitblue}{RGB}{20,134,225}
  \definecolor{gmitred}{RGB}{220,20,60}
  \definecolor{gmitgrey}{RGB}{67,67,67}

  % Set minted up.
  \usemintedstyle{manni}
  \setminted{baselinestretch=1.2, bgcolor=gmitgrey!10}

  % Change the displayed name of the references section.
  \renewcommand{\refname}{\selectfont\large References} 

  % Heading
  \title{\vspace{-20mm}Assessment}
  \author{Machine Learning and Statistics, Winter 21/22}
  \date{Due: last commit on or before December 19\textsuperscript{th}, 2021\vspace{-6mm}}

\begin{document}
  
  \maketitle

  These are the instructions for the assessment of Machine Learning and Statistics in Winter 2021/2022.
  The assessment is worth 100\% of the marks for the module.
  Please read the \emph{Using git for assessments}~\cite{usinggit} document on the Moodle page which applies here.
  As always, you must also follow the code of student conduct and the policy on plagiarism~\cite{gmitqaf}.

  \section*{Instructions}
  
  The purpose of this assessment is to ensure that you have achieved the learning outcomes of the module while also providing you with sample work to show prospective employers.
  The overall assessment is split into three components, as follows.
  The percentages beside each bullet point indicate the overall weighting of that item in your overall mark.
  However, the examiners' overall impression of your submission may override the individual weightings where deemed appropriate.

  \subsubsection*{GitHub Repository}
  Create a GitHub repository containing two Jupyter notebooks -- these are described further down.
  The repository should contain the following:
  \begin{description}
    \item[10\%] A clear and informative \texttt{README.md} explaining what is in the repository, why the repository exists, and how to run the notebooks.
    \item[10\%] A \texttt{requirements.txt} file file that enables someone to quickly run your notebooks with minimal configuration. You should also include any other required files such as data files and image files.
  \end{description}

  \subsubsection*{Scikit-Learn Jupyter Notebook}
  Include a Jupyter notebook called \texttt{scikit-learn.ipynb} that contains the following.
  \begin{description}
    \item[10\%] A clear and concise overview of the \texttt{scikit-learn} Python library~\cite{scikit-learn}.
    \item[20\%] Demonstrations of three interesting \texttt{scikit-learn} algorithms. You may choose these yourself, based on what is covered in class or otherwise. Note that the demonstrations are at your discretion -- you may choose to have an overall spread of examples across the library or pick a particular part that you find interesting.
    \item[10\%] Appropriate plots and other visualisations to enhance your notebook for viewers.
  \end{description}
  
  \subsubsection*{Scipy Stats Jupyter Notebook}
  Include a Jupyter notebook called \texttt{scipy-stats.ipynb} that contains the following.
  \begin{description}
    \item[10\%] A clear and concise overview of the \texttt{scipy.stats} Python library~\cite{scipy-stats}.
    \item[20\%] An example hypothesis test using ANOVA. You should find a data set on which it is appropriate to use ANOVA, ensure the assumptions underlying ANOVA are met, and then perform and display the results of your ANOVA using \texttt{scipy.stats}.
    \item[10\%] Appropriate plots and other visualisations to enhance your notebook for viewers.
  \end{description}

  \section*{More information about marking}
    The following four categories will be used to consider each componenet of your submission.
    It is important that your submission provides direct evidence for each category.
    For instance, your commit history should demonstrate and provide evidence that you had a pragmatic attitude to completing the assessment.
    Likewise, your submission should have references in it to demonstrate that you considered the literature and the work of others.
  
    \subsubsection*{Research}
    Evidence of research performed on topic; submission based on referenced literature, particularly academic literature; evidence of understanding of the documentation for any software or libraries used.
    \subsubsection*{Development}
    Environment can be set up as described; code works without tweaking and as described; code is efficient, clean, and clear; evidence of consideration of standards and conventions appropriate to code of this kind.
    \subsubsection*{Consistency}
    Evidence of planning and project management; pragmatic attitude to work as evidenced by well-considered commit history; commits are of a reasonable size; consideration of how commit history will be perceived by others.
    \subsubsection*{Documentation}
    Clear documentation of how to create an environment in which any code will run, how to prepare the code for running, how to run the code including setting any options or flags, and what to expect upon running the code. Concise descriptions of code in comments and README.
    
  \bibliographystyle{IEEEtran}
  \bibliography{bibliography}
  
\end{document}
